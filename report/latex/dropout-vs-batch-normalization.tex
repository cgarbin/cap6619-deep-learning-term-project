\documentclass[conference]{IEEEtran}
% \IEEEoverridecommandlockouts
% The preceding line is only needed to identify funding in the first footnote. If that is unneeded, please comment it out.
\usepackage{amsmath,amssymb,amsfonts}
\usepackage{algorithmic}
\usepackage{graphicx}
\usepackage{textcomp}
\usepackage{xcolor}
\usepackage{subfiles}

\usepackage[colorinlistoftodos]{todonotes}

% For side-by-side figures
% Source: https://tex.stackexchange.com/questions/271518/multiple-panel-figure-with-figures-side-by-side
% \usepackage{caption} -- already included by subcaption
\usepackage{subcaption}
\usepackage{graphicx}

% For multi-line table headers
% Source: https://tex.stackexchange.com/questions/248053/a-header-in-a-table-with-two-lines
\usepackage{makecell}

% For source code listing
\usepackage{listings}

% For inline hyperlinks
\usepackage{hyperref}

\usepackage[
backend=biber,
style=alphabetic,
sorting=ynt
]{biblatex}
\addbibresource{dropout-vs-batch-normalization.bib}

\def\BibTeX{{\rm B\kern-.05em{\sc i\kern-.025em b}\kern-.08em
    T\kern-.1667em\lower.7ex\hbox{E}\kern-.125emX}}

\begin{document}

\title{Dropout \textit{vs.} Batch Normalization: An Empirical Study of Their Impact to Deep Learning}

\author{\IEEEauthorblockN{Christian Garbin}
\IEEEauthorblockA{\textit{College of Engineering and Computer Science} \\
\textit{Florida Atlantic University}\\
Boca Raton, Florida, USA \\
CAP-6619 Deep Learning, Fall 2018\\
Term Project}}

\maketitle

\begin{abstract}
\subfile{sections/1-abstract}
\end{abstract}

\begin{IEEEkeywords}
Deep learning, overfitting, dropout, batch normalization, optimization, regularization
\end{IEEEkeywords}

\section{Introduction}
\subfile{sections/2-introduction}

\section{Related Work}
\subfile{sections/3-related-work}

\section{Empirical Study Framework and Designs}
\subfile{sections/4-framework-and-design}

\section{Experiments and Settings}\label{sec:experiments}
\subfile{sections/5-experiments}

\section{Results and Analysis}\label{sec:results}
\subfile{sections/6-results-and-analysis}

\section{Conclusions}
\subfile{sections/7-conclusions}


\printbibliography

\section{Appendix}

\subsection{Picture attribution}

Picture to illustrate machine learning types is from Guyspeakwell at English Wikipedia, provided under the \href{https://creativecommons.org/licenses/by/3.0/deed.en}{Attribution 3.0 Unported} license. Color-coded rectangles were added to the picture to illustrate image segmentation and classification.

\subsection{Source code used for experiments}

The source code used in the experiments is available in Github at \url{https://github.com/fau-masters-collected-works-cgarbin/cap6619-deep-learning-term-project}.

\subsection{How network model pictures were generated}

Pictures showing network models were generated in Keras with \verb|plot_model(...)|, e.g. 

\begin{lstlisting}[language=Python]
plot_model(model, to_file="modelplot.png",
           show_layer_names=False,
           show_shapes=True)
\end{lstlisting}

The pictures were created right after adding the layers to the models and calling \verb|model.compile()|. The pictures reflect the model exactly as they were trained and tested.

\end{document}
